
%%% Informe TP final de SO1


\documentclass[a4paper]{report}
\usepackage{amsmath}
\usepackage{graphicx}
\usepackage{float}
\renewcommand{\baselinestretch}{1}
\usepackage[margin=2cm]{geometry}
\usepackage{amssymb}
\usepackage{fancyhdr}
\usepackage[utf8]{inputenc}
\renewcommand{\familydefault}{\sfdefault}
\pagestyle{fancy}
\lhead{Trabajo pr\'actico final}
\rhead{Bruno Sotelo}

\newcommand{\femph}[1]{\textbf{\textit{#1}}}

\setlength{\parindent}{0in}

\begin{document}


\begin{titlepage}
\centering
\begin{figure}[H]
    \begin{center}
        \includegraphics[width=2cm,height=2cm]{UNR_logo.jpg}
    \end{center}
\end{figure}
{\scshape\large Facultad de Ciencias Exactas, Ingenier\'ia y Agrimensura\\*
                 Licenciatura en Ciencias de la Computaci\'on\par}
\vspace{5cm}
{\scshape\LARGE Sistemas Operativos 1 \par}
{\huge\bfseries Trabajo Pr\'actico Final \par}
\vspace{3cm}
{\Large Sotelo, Bruno\par}
\vfill
{\large 23 / 01 / 2018 \par}
\end{titlepage}


\section*{Explicación en general del servidor}

El servidor de juegos fue implementado en 4 archivos:
\begin{description}
	\item[interface] Donde se encuentran las funciones que manejan
		las conexiones tcp del servidor, como \textit{psocket}, 
		\textit{updts\_sender} y \textit{dispatcher}.
	\item[commands] En el cual se encuentran \textit{pcommand} y 
		\textit{player\_control}.
	\item[services] Con funciones que actuaran como servicios
		del servidor, por ejemplo \textit{games\_names} posee información
		de todas las partidas en el servidor. Otras funciones importantes
		son \textit{balance, pstat, players\_names} y \textit{games\_msgs}.
	\item[tateti] En donde se implementa el juego de Ta-te-ti en sí, el
		cual se divide en cuatro fases.
\end{description}
\subsection*{interface.erl}
Para ejecutar el servidor se utiliza 
\femph{start\_server(Port,Node)}, donde \textit{Port} es el puerto
en el cual estará escuchando el servidor y \textit{Node} es otro nodo
de Erlang que esté ejecutando este c\'odigo, con el cual se
conectará para formar un servicio distribuido; si no existe otro nodo o no se
desea formar un cluster, se puede utilizar el átomo \textit{false} en este
argumento. Esta función, además de conectar a un cluster de nodos, comenzará
los servicios que usará el servidor (funciones de \textit{services.erl}), y 
los registrará con un átomo. Luego llamará a
\femph{start\_listen} para comenzar a escuchar en el puerto dado.\\
La función \femph{dispatcher}, como fue indicado, acepta nuevas conexiones y
crea un \femph{psocket} para cada cliente, el cual recibe todos sus 
mensajes, además de contestar a los comandos (con OK o ERROR). Para enviar
las actualizaciones de una partida se utiliza \femph{updts\_sender}, cada
mensaje se envía sólo si el anterior ya fue contestado por el cliente.

\subsection*{commands.erl}
Cada comando recibido es procesado por \femph{pcommand}. En varios
comandos el trabajo es simplemente enviar un mensaje a algún servicio y 
esperar la respuesta. En \textit{commands.erl} también encontramos a 
\femph{player\_control}, una función que será ejecutada por un proceso por
cada cliente que haya en el servidor. Esta función mantiene dos listas para
las partidas que el cliente está jugando y observando, una para cada caso.
También mantiene el nombre que el cliente registró (o el átomo 
\textit{undefined} si no lo hizo), y se encarga de redireccionar los mensajes
de UPD a \textit{updts\_sender}. Estos mensajes no se transmiten directamente
a esa función para no cargar con más variables al juego de Ta-te-ti y
mantener su código más limpio. Así el juego sólo necesita mantener los ID
de los procesos \textit{player\_control} de cada cliente para comunicar
actualizaciones. Aprovechando esto, se pueden usar estos ID como
identificadores de cada cliente, ya que serán únicos en todo el cluster de
servidores.

\subsection*{services.erl}
Entre las funciones de \textit{services.erl} encontramos a \femph{balance} y
\femph{pstat}, que juntas se encargan de comunicar en el cluster las cargas
de cada nodo y decidir qué nodo debe atender el próximo comando entrante de
un cliente. También se encuentra \femph{players\_names} que se encarga de
mantener los nombres registrados en el nodo, y, por medio del protocolo 
descrito en la función, asegurarse de que un nuevo cliente no registre un
nombre que ya se encuentre en el cluster. \\
Otro servicio en \textit{services.erl} es \femph{games\_names}. Se encarga
de mantener la lista de juegos que se están llevando a cabo en el nodo.
Cuando un cliente requiere los juegos que puede ver u observar, se manda un
mensaje con \textit{get\_games} a este servicio en todos los nodos, con lo
cual se recuperan todos los juegos accesibles. Además de mantener los
nombres de las partidas, mantiene sus estados, lleno o no, para que el
cliente pueda conocer a cuál puede acceder. Este servicio participa en la
creación de un juego nuevo, y una partida puede, a través de un mensaje,
cambiar su estado y borrarse de la lista al finalizar. \\
Un jugador puede estar conectado a un nodo distinto al que mantiene el juego
al que accedió, por lo tanto surgió como idea utilizar strings de la forma
"\{Nombre-Nodo\}+\{PID-Local\}" para identificar partidas. Esto permite que, dado
un comando destinado a una partida, el mensaje pueda ser dirigido utilizando
sólo el ID de esta. La entrega del mensaje se realiza por medio del servicio
\femph{games\_msgs}. Este servicio recibe mensajes de la función
\textit{pcommand}, y los reenvía al mismo servicio en el nodo donde se
encuentra la partida, el cual puede dirigir el mensaje a la partida teniendo
su ID local. S\'olo los mensajes \textit{ACC} deben dirigirse a
\textit{games\_names} para cerciorarse de que la partida no est\'e llena.

\subsection*{tateti.erl}
Cuando un cliente pide una nueva partida, se crea un proceso que ejecuta
\femph{ttt\_phase1}. En esa funci\'on se consulta el nombre del cliente
y el GID que le ser\'a asignado al juego, y ambos son devueltos como
respuesta al cliente. Luego el juego pasa a la segunda fase
(\femph{ttt\_phase2}), donde se espera al segundo jugador. Si el jugador
no registr\'o un nombre se lo rechaza y se sigue esperando. Si no, se
prepara todo para la partida y se pasa a la fase 3 (\femph{ttt\_phase3})
donde se da el juego en s\'i. En cada iteraci\'on de esta fase se reciben
los mensajes para observar o dejar de observar, se env\'ia una actualizaci\'on
al jugador que debe jugar y a los observadores, y se espera una jugada del
cliente requerido. Por cada jugada se analiza si el jugador gan\'o, si hubo
un empate, o si la jugada fue inv\'alida (en este caso el jugador pierde su
turno). \\
En el caso de que alg\'un jugador gane, abandone, o en caso de empate, se
termina el juego con un \'ultimo \textit{UPD} a los jugadores indicados.
Este se manda desde la funci\'on \femph{ttt\_end}, y menciona la raz\'on
por la cual termin\'o el juego, y el ganador en caso de haberlo.

\section*{Sobre el cliente}
El cliente implementado en \textit{client.erl} es un simple programa de
consola, con el cual se puede dialogar con el servidor a través de algunos
comandos, que son una leve abstracción de los que puede manejar el servidor.
Entre otras cosas, lleva los ID de comandos automáticamente, puede responder
a las actualizaciones y no necesita que se tipeen los ID de juegos. \\
Aunque el cliente podría, debido a la implementación del servidor, jugar y
observar varios juegos a la vez, en este cliente sólo se podrá acceder a 
un juego hasta que termine (o abandone). \\
El cliente se compone de cinco procesos cooperativos:
\begin{description}
	\item[connection] Recibe todos los paquetes TCP del servidor y hace un
	pequeño procesamiento de cada uno para saber a dónde dirigirlo.
	\item[reader] Es el proceso que lee la entrada del cliente y la entrega
	a \textit{cmds\_send}.
	\item[cmds\_send] Procesa los comandos que entra el cliente y los 
	transforma a comandos que acepta el servidor para mandárselos.
	\item[games] Lleva la última lista de juegos conocida. Esta se actualiza
	cada vez que el cliente emite el comando 'juegos'.
	\item[updt\_recv] Recibe las actualizaciones de los juegos, y se encarga
	de imprimir mensajes una vez entrado al juego, como el tablero.
\end{description}

\section*{Posibles mejoras al servidor}
\begin{itemize}
	\item Presentar otros ID para los juegos, ya que los actuales revelan
	mucha información a los clientes, lo cual se puede utilizar para
	vulnerar el servidor. Entre las soluciones están: encriptar los ID,
	implementar un nuevo servicio que mantenga una biyección entre estos
	IDs y los que se presentarían públicamente, cambiar el modelo de IDs
	a otro distinto, etc.
	\item Estructurarlo mejor para llegar a un modelo que acepte no sólo
	el juego Ta-te-ti sino otros juegos basados en turnos (TBS).
	\item Implementar un sistema de chat.
	\item Agregar una base de datos que lleve registro de jugadores, 
	partidas, etc.
	\item Implementar un cliente con interfaz gráfica.
	\item Agregar seguridad contra diferentes tipos de ataques por parte
	de clientes, y mejorar la robustez del código, así el servidor podría
	escalar mejor mientras se agreguen más jugadores.
\end{itemize}



\end{document}

